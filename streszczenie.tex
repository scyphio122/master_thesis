\thispagestyle{empty}
\noindent {\Large\textbf{Streszczenie}}\\

\noindent {\Large\textbf{Projekt urządzenia do lokalizacji pojazdów w trybie on i off-line}}\\

\begin{singlespacing}

Celem pracy było zaprojektowanie, wykonanie i oprogramowanie urządzenia stanowiącego dodatkowe zabezpieczenie pojazdu na wypadek kradzieży, w postaci lokalizatora wykorzystującego system GNSS (\textit{ang. Global Navigation Satellite System}) oraz GSM (\textit{ang. Global System for Mobile Communications}), zdolnego do analizy stylu jazdy kierowcy, a także całego systemu informatycznego, który pozwoliłby na obsłużenie pozyskanych danych. Założeniem jest, aby moduł elektroniczny mógł być zasilany z instalacji elektrycznej pojazdu, lecz poza tym był od niego całkowicie niezależny.\\

System opisywany w pracy przeznaczony jest głównie dla trzech grup osób. Pierwszą z nich są właściciele firm posiadających flotę pojazdów, prowadzonych przez pracowników. Umożliwia im zdalny podgląd tras przebywanych przez zatrudnione osoby, a także syntetyczną informację o sposobie jazdy kierowcy. Drugą kategorię stanowią osoby zainteresowane możliwością zdalnego zlokalizowania pojazdu w przypadku kradzieży, dzięki wykorzystaniu mechanizmu alarmowania poprzez wiadomości SMS. Ostatnia grupa to osoby, które chciałyby poprawić swoje umiejętności ekologicznego i bezpiecznego prowadzenia pojazdów, na podstawie dynamicznej oceny wydawanej przez lokalizator. \\

W skład systemu wchodzą dwa urządzenia elektroniczne. Jedno z nich służy do lokalizacji pojazdu, ciągłej oceny stylu jazdy, wysyłania danych na serwer, wykorzystując sieć GSM oraz awaryjnego powiadamiania właściciela w przypadku wykrycia kradzieży. Drugie urządzenie ma za zadanie autoryzować ruch pojazdu i deaktywować funkcję alarmu. Komunikacja pomiędzy obydwoma urządzeniami odbywa się zabezpieczonym kanałem poprzez Bluetooth Low Energy z wykorzystaniem algorytmu szyfrowania AES128. \\

Kolejnym elementem systemu jest aplikacja serwerowa napisana w języku C++. Jej celem jest odbieranie danych transmitowanych przez urządzenie, przetworzenie ich i zapisanie w bazie danych. Ponadto, jest ona odpowiedzialna za obsługę zapytań HTTP pochodzących ze strony internetowej, która stanowi ostatni moduł pracy. Jest ona wykorzystywana w celu rejestracji i logowania użytkowników do systemu, dodawania urządzeń, a także wyświetlania danych o przebytych trasach oraz ocenach stylu jazdy i ich wizualizacji na mapach od firmy Google. \\

W ramach pracy przeprowadzono również badania nad sposobem analizy i oceny stylu jazdy kierowcy, w oparciu o informacje dotyczące przyspieszenia i jego zmian pochodzące z akcelerometru. Badania zaowocowały przedstawieniem autorskiego algorytmu wykrywania agresywnego sposobu prowadzenia pojazdów. Wykonane samodzielnie testy drogowe wykazały jego skuteczność. \\

\textbf{Słowa kluczowe: }analiza stylu jazdy, Bluetooth Low Energy, GPS, GSM, lokalizacja, pojazd, samochód, szyfrowanie AES

\end{singlespacing}

\cleardoublepage
\thispagestyle{empty}

\noindent {\Large\textbf{Abstract}}\\


\noindent {\Large\textbf{The project of the device localizing vehicles on and off-line}}\\

\begin{singlespacing}


The main goal of this thesis is to design, build and program the device which is additional security module in case of vehicle's theft in the form of a GNSS (\textit{Global Navigation Satellite System}) and GSM (\textit{Global System for Mobile Communication}) system based locator with ability to analyze driving style of the vehicle user. Moreover, the aim was to create entire IT system for handling the gathered data. The assumption is that the electronic module is able to be powered by the vehicle's electrical system and yet still be completely independent. \\

The system described in the thesis is dedicated for three groups of recipients. The first one are owners of the companies that have vehicle's fleet which is used by the employees. It gives them the ability to remotely display the routes traveled by the employees along with the brief information about the driving style. The second category is the group of people interested in remote vehicle locating in case of the theft, thanks to the embedded alarming mechanism utilizing the SMS messages. The last group is made of customers who want to enhance their ecological driving skills based on the dynamically calculated driving style assessment. \\

The system consists of two electronic devices. The first one is used to locate vehicles,  continuously calculate driving style assessment, transmit the data on the web server via GSM network and notify the vehicle owner about the car theft. The function of the second device is to authorize the vehicle movement and deactivate the alarm feature. The communication between the devices relies on the secured channel based on the Bluetooth Low Energy protocol with usage of AES128 encryption algorithm. \\

The next part of the system is the server application written in C++ programming language. Its aim is to receive the data sent by the locator device, process it and store it in the database. Moreover, it is responsible for handling HTTP requests incoming from the website designed in the thesis. The website is used in order to register and authorize users, add devices and visualize the data regarding the traveled tracks and their assessments using the Google Map API. \\

Also, within this thesis the author conducted the research about the way of driving style analysis and assessment, based on the information about the acceleration and its changes received from the accelerator module used in the locator device. The experiments resulted in the proposition of the innovative algorithm to evaluate the driving style. Self-conducted road tests confirmed its efficiency. \\

\flushbottom
\textbf{Key words: }Bluetooth Low Energy, car, driving style analysis, GPS, GSM, localization, vehicle, AES encryption
\end{singlespacing}
