\thispagestyle{empty}
\noindent {\Large\textbf{Streszczenie}}\\

\noindent {\Large\textbf{Projekt urządzenia do lokalizacji pojazdów w trybie on i off-line}}\\

\begin{singlespacing}

Celem pracy było zaprojektowanie, wykonanie i oprogramowanie urządzenia stanowiącego dodatkowe zabezpieczenie antykradzieżowe pojazdu w postaci lokalizatora wykorzystującego system GNSS (\textit{ang. Global Navigation Satellite System}) oraz GSM (\textit{ang. Global System for Mobile Communications}), zdolnego do analizy stylu jazdy kierowcy, a także całego systemu informatycznego, który pozwoliłby na obsłużenie pozyskanych danych. Założeniem jest, aby projekt był niezależny od instalacji i czujników dostępnych w samochodzie.\\

System opisywany w pracy przeznaczony jest głównie dla trzech grup osób. Pierwszą z nich są właściciele firm posiadających flotę pojazdów, prowadzonych przez pracowników. Umożliwia im zdalny podgląd tras przebywanych przez zatrudnione osoby, a także syntetyczną informację o sposobie jazdy kierowcy. Drugą kategorię stanowią osoby zainteresowane możliwością zdalnego zlokalizowania pojazdu w przypadku kradzieży, dzięki wykorzystaniu mechanizmu alarmowania poprzez wiadomości SMS. Ostatnia grupa to osoby, które chciałyby poprawić swoje umiejętności ekologicznego i bezpiecznego prowadzenia pojazdów, na podstawie dynamicznej oceny wydawanej przez lokalizator. \\

W skład systemu wchodzą dwa urządzenia elektroniczne. Jedno z nich służy do lokalizacji pojazdu, ciągłej oceny stylu jazdy, wysyłania danych na serwer wykorzystując sieć GSM oraz awaryjnego powiadamiania właściciela w przypadku wykrycia kradzieży. Drugie urządzenie ma za zadanie autoryzować ruch pojazdu i deaktywować funkcję alarmu. Komunikacja pomiędzy obydwoma urządzeniami odbywa się zabezpieczonym kanałem poprzez Bluetooth Low Energy z wykorzystaniem algorytmu szyfrowania AES128. \\

Kolejnym elementem systemu jest aplikacja serwerowa napisana w języku C++. Jej celem jest odbieranie danych transmitowanych przez urządzenia, przetworzenie ich i zapisanie w bazie danych. Ponadto, jest ona odpowiedzialna za obsługę zapytań HTTP pochodzących ze strony internetowej stanowiący ostatni moduł pracy. Jest ona wykorzystywana w celu rejestracji i logowania użytkowników do systemu, dodawania urządzeń, a także wyświetlania danych o przebytych trasach oraz ocenach stylu jazdy i ich wizualizacji na mapach od firmy Google. \\

W ramach pracy przeprowadzono również badania nad sposobem analizy i oceny stylu jazdy kierowcy, w oparciu o informacje dotyczące przyspieszenia i jego zmian pochodzące z akcelerometru. Badania zaowocowały przedstawieniem autorskiego algorytmu wykrywania agresywnego sposobu prowadzenia pojazdów. Wykonane samodzielnie testy drogowe wykazały jego skuteczność. \\

\textbf{Słowa kluczowe: }analiza stylu jazdy, Bluetooth Low Energy, GPS, GSM, lokalizacja, pojazd, samochód, szyfrowanie AES

\end{singlespacing}

\cleardoublepage
\thispagestyle{empty}

\noindent {\Large\textbf{Abstract}}\\


\noindent {\Large\textbf{The project of the device localizing vehicles on and off-line}}\\

\begin{singlespacing}


The main goal of this thesis is to design and build the entire IT system which would be used to gather information about vehicles localization and perform dynamic driving style assessment of the driver. It bases on the dedicated electronic device which utilizes systems like GPS for localization, GSM for long range communication or Bluetooth Low Energy and Near Field Communication for mid and short range data exchange. Gathered data is transmitted nearly in the real time to the database server from which it can be displayed via designed as a part of the system website. Moreover, the device has additional function to alarm the car owner about its location in case of theft.\\

\flushbottom
\textbf{Key words: }Bluetooth Low Energy, car, driving style analysis, GPS, GSM, localization, vehicle, AES encryption
\end{singlespacing}
