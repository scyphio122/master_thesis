

\chapter*{Streszczenie}

\flushbottom

\noindent {\Large\textbf{Projekt urządzenia do lokalizacji pojazdów w trybie on i off-line}}

\begin{singlespacing}

Praca ta ma na celu wykonanie systemu informatycznego, służącego do gromadzenia danych o lokalizacji pojazdów, który dodatkowo umożliwiałby dynamiczną analizę i ocenę stylu jazdy kierowców. Bazuje ona na samodzielnie zaprojektowanym, dedykowanym urządzeniu lokalizującym, które wykorzystuje system GPS do lokalizacji oraz system GSM do komunikacji bezprzewodowej dużego zasięgu, a także interfejsy Bluetooth Low Energy i Near Field Communication do komunikacji średniego i bliskiego zasięgu. 

Dane są zbierane cyklicznie w niewielkich odstępach czasowych, a następnie wysyłane praktycznie w czasie rzeczywistym do aplikacji serwerowej, które dokonuje ich przetworzenia i umieszcza w bazie danych. W ramach pracy zrealizowano również projekt strony internetowej, umożliwiającej zdalny podgląd danych w każdym momencie. 

Ponadto, urządzenie stanowi dodatkowy element zabezpieczający pojazd w razie kradzieży. Ma on bowiem zdolność wykrywania nieautoryzowanego uruchomienia pojazdu, które wyzwala zadanie cyklicznego powiadamiania właściciela o jego lokalizacji. Dokonywane jest to poprzez wysłanie na telefon  wiadomości SMS. 

W ramach pracy przeprowadzono również badania nad sposobem analizy i oceny stylu jazdy kierowcy, które miały na celu umożliwienie właścicielom oraz managerom firm posiadających własną flotę pojazdów gromadzenie pełniejszych informacji na temat sposobu ich użytkowania. Badania zaowocowały przedstawieniem autorskiego algorytmu wykrywania agresywnego sposobu prowadzenia pojazdów.

\textbf{Słowa kluczowe: }analiza stylu jazdy, Bluetooth Low Energy, GPS, GSM, lokalizacja, pojazd, samochód, szyfrowanie komunikacji

\end{singlespacing}


\chapter*{Abstract}

\flushbottom

\noindent {\Large\textbf{The project of the device localizing vehicles on and off-line}}

\begin{singlespacing}


The main goal of this thesis is to design and build the entire IT system which would be used to gather information about vehicles localization and perform dynamic driving style assessment of the driver. It bases on the dedicated electronic device which utilizes systems like GPS for localization, GSM for long range communication or Bluetooth Low Energy and Near Field Communication for mid and short range data exchange. Gathered data is transmitted nearly in the real time to the database server from which it can be displayed via designed as a part of the system website. Moreover, the device has additional function to alarm the car owner about its location in case of theft.

\flushbottom
\textbf{Key words: }Bluetooth Low Energy, car, driving style analysis, GPS, GSM, localization, vehicle
\end{singlespacing}
