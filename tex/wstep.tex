\chapter{Wstęp}
\label{ch:wstep}

Problem lokalizacji robota w pomieszczeniu zamkniętym jest w ostatnich latach często rozważany. Jako że nawigacja satelitarna (GPS, GLONASS) jest niedostępna w pomieszczeniach zamkiętych, konieczne jest opracowanie innych metod lokalizowania robota. Do tych metod należą m. in:
\begin{itemize}
 \item lokalizacja w oparciu o wizualne znaczniki i system ich rozpoznawania
 \item lokalizacja na podstawie stereowizji
 \item odometria
 \item lokalizacja na podstawie odległości od znaczników (radiowych, akustycznych itp)
\end{itemize}

Przedmiotem niniejszej pracy jest zaprojektowanie oprogramowania do znacznika radiowego i odbiornika, pozwalającego na wyznaczanie odległości odbiornika do znacznika na podstawie parametru RSSI (Received Signal Strength Indication). Parametr RSSI określa moc odbieranego sygnału radiowego.

Takie znaczniki mogą zostać rozmieszczone w środowisku pracy robota, z kolei robot może zostać niskim kosztem wyposażony w odbiornik radiowy \cite{localization}. Dysponując mapą rozmieszczenia znaczników w pomieszczeniu oraz informacją o odleglościach pomiędzy robotem a poszczególnymi znacznikami, można wyznaczać pozycję robota. 

Ze względu na niski koszt sprzętu i łatwość implementacji, do implementacji rozwiązania wybrano protokół Bluetooth Low Energy.
