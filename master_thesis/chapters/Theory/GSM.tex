\section{System GSM}
\label{GSM}

\subsection{Opis sieci GSM}

System GSM (\textit{ ang. \textbf{G}lobal \textbf{S}ystem for \textbf{M}obile Communication}) jest obecnie najpowszechniej wykorzystywanym systemem służącym do komunikacji bezprzewodowej dalekiego zasięgu. System ten wykorzystywany jest do przesyłania głosu oraz serwisów danych. Pomysł na stworzenie sieci umożliwiającej komunikację głosową wyłonił się we wczesnych latach 70. ubiegłego wieku z opracowywanego w siedzibie Bell Laboratories mobilnej sieci radiowej. Jednakże dopiero dwanaście lat później, w 1982 roku powstał oficjalny komitet normalizacyjny nazwany \textit{Groupe Spécial Mobile}, którego zadaniem było utworzenie jednolitego, otwartego standardu dla telefonii komórkowej. 

Pierwotna wersja standardu działała w paśmie 900 MHz (880 - 960 MHz) i umożliwiała jedynie transmisję głosową. Jego kolejna wersja została opublikowana w 1990r. i definiowała ona dodatkowe pasmo 1800 MHz (1710 - 1880 MHz). Ponadto, umożliwiała przesyłanie krótkich wiadomości SMS (\textit{ang. \textbf{S}hort \textbf{M}essage \textbf{S}ystem}), a także faxu czy transmisję danych. Dalsze prace nad systemem wprowadziły do standardu techniki zwiększające przepustowość transmisji (maksymalna prędkość odbioru - 57.6 kb/s, maksymalna prędkość nadawania - 14.5 kb/s oraz 30 - 80 kb/s przy transmisji GPRS) oraz mechanizm przesyłania danych w pakietach (GPRS \textit{ang. \textbf{G}eneral \textbf{P}acket \textbf{R}adio \textbf{S}ervice}). 

Pomimo pojawienia się na świecie nowszych rozwiązań, takich jak sieci UMTS i LTE, ze względu na ogromną popularność, architektura sieci GSM wciąż jest rozwijana. 


System GSM umożliwia skorzystanie z następujących usług:
\begin{itemize}
	\item Połączenia głosowe - Stanowią one sztandarowe zastosowanie dla sieci GSM. Jej standard definiuje kodek GSM, który służy do zamiany głosu (napięciowego sygnału analogowego) na postać cyfrową, transmitowaną poprzez sieć.
	\item Transmisja danych - Umożliwia dostęp do internetu z urządzenia GSM, a także korzystanie z transmisji strumieniowej.
	\item Wiadomości tekstowe i multimedialne - Usługa przesyłania krótkich wiadomości tekstowych, o długości do 160 znaków, pod warunkiem korzystania jedynie z alfabetu łacińskiego. W przypadku stosowania znaków diakrytycznych maksymalny rozmiar wiadomości spada do 70 znaków. Wiadomości multimedialne (inaczej MMS), umożliwiają przesyłanie zdjęć, filmów czy dźwięków. Ich rozmiar maksymalny jest uzależniony od ograniczeń telefonu oraz operatora.
\end{itemize}

