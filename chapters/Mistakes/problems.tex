\chapter{Problemy i ich rozwiązania}
\label{problems_and_solutions}

Nieodłączną częścią każdego projektu informatycznego i elektronicznego są problemy związane z uruchomieniem. Nie inaczej było w przypadku niniejszej pracy i opisywanego w nim systemu. W trakcie jego rozwoju napotkano szereg różnorakich komplikacji, z którymi należało sobie poradzić. W tym rozdziale dokonano wyboru i krótkiego opisu wraz z solucją trzech najciekawszych z nich.


\section{Zawieszanie urządzenia lokalizującego}

Pierwszy z wybranych problemów dotyczył pierwszych momentów uruchomienia lokalizatora. Polegał on na tym, że mimo poprawnego zaprogramowania mikrokontrolera i działającej poprawnie apkilacji z podłączonym do niej debuggerem, po jego odłączeniu i dokonaniu resetu urządzenia urządzenie zawieszało się gdzieś we wstępnej fazie inicjalizacji. Można to było wywnioskować po fakcie, iż zapalała się dioda LED, sterowana programowo, lecz moduł GSM nigdy nie rozpoczynał swej inicjalizacji. Po wczytaniu się w notę katalogową producenta mikrokontrolera okazało się, iż problem stanowił wykorzystywany do odmierzania czasu, wbudowany w rdzeń przez firmę ARM zegar systemowy SysTick. Powodem jego błędnej pracy było każdorazowe wejście mikrokontrolera w tryb oszczędzania energii w trakcie oczekiwania na upłynięcie określonej ilości czasu (\textit{ang. delay}). Powoduje on bowiem zatrzymanie sygnału taktującego rdzeń mikrokontrolera, a tym samym - SysTick'a, wykorzystującego zegar systemowy. Efektem jest wówczas praca w pętli i oczekiwanie na upłynięcie losowej ilości czasu. Losowość ta jest wywołana faktem, iż rdzeń wybudzany jest dowolnym przerwaniem na pewien krótki czas, a zatem SysTick "ożywał" na krótko by znów ulec po chwili zatrzymaniu. Co ciekawe, problem ten nie występował w trakcie pracy z debuggerem, ponieważ wymusza on ciągłą pracę zegara systemowego niezależnie od trybu oszczędzania energii. Rozwiązaniem było zastąpienie SysTick'a innym zegarem - wbudowanym w mikrokontroler bardzo energooszczędnym zegarem RTC (\textit{ang. \textbf{R}eal \textbf{T}ime \textbf{C}lock}).

\section{Brak danych z GPS}

Kolejny problem związany był typowo z elektroniką. Polegał on na tym, że mimo włączania modułu GPS zgodnie z notą katalogową producenta chip'a GSM i GPS - poprzez komendę AT wysyłaną poprzez interfejs UART do pośredniczącego modułu GSM, moduł GPS nie odpowiadał. Ponadto, nie był responsywny na żadną inną komendę. Problem ten wynikał z błędnego zrozumienia noty katalogowej od producenta, dotyczącej zasilania modułu GPS. Posiada on bowiem osobny pin, do którego powinno zostać dostarczone zasilanie (\textit{$GNSS_VCC$}) oraz pin (\textit{$GNSS_VCC_EN$}), który przyjmuje stan wysoki (napięcie o wartości ok. 3.3 V) gdy GPS jest włączony oraz stan niski (napięcie bliskie 0 V) gdy jest wyłączony. Pin \textit{$GNSS_VCC_EN$} reagował poprawnie - zmieniał stan po włączeniu modułu GPS komendą AT. Błędnym rozumowaniem okazało się założenie, że jest to jedynie fizyczny wskaźnik statusu zasilania modułu GPS. W praktyce, zamysłem producenta chip'a było, aby pin ten wysterowywał tranzystor bądź zewnętrzne źródło zasilania, w celu podania napięcia na pin \textit{$GNSS_VCC$}. Wynika to z faktu występowania sztywnych zależności czasowych w wewnętrznej komunikacji między modułami GSM i GPS. Rozwiązaniem tego problemu było przeprojektowanie zasilania modułu GPS zgodnie z założeniem producenta chip'a.

