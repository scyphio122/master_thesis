\section{Urządzenie lokalizujące}
\lstset{language=C}

Urządzenie lokalizujące realizuje następujące funkcje: 

\begin{itemize}
\item wygenerowanie głównego klucza szyfrującego i sparowanie z urządzeniem deaktywującym,
\item zapewnienie bezpiecznego kanału komunikacji w trakcie połączenia z \textit{Key Tag'iem},
\item wykrywanie ruchu pojazdu,
\item komunikacja z urządzeniem deaktywującym, w celu podjęcia próby deaktywacji alarmu,
\item alarmowe powiadamianie właściciela w przypadku nieautoryzowanego przemieszczenia pojazdu,
\item pobieranie próbek lokalizacji, prędkości, przyspieszenia, a  także aktualnego kursu (azymutu) i innych parametrów po wykryciu ruchu oraz ich cykliczne wysyłanie na zdalny serwer danych,
\item analiza stylu jazdy kierowcy,
\item wysyłanie poprzez SMS lokalizacji pojazdu na żądanie użytkownika.
\end{itemize}

\clearpage
Duża liczba zadań realizowanych przez mikrokontroler sterujący oraz konieczność wywoływania ich po ściśle określonej sekwencji czasowej spowodowała, że niezbędnym stało się wprowadzenie modułu planisty. Stanowi on bardzo prostą funkcjonalność, bez możliwości wywłaszczania zadań i zmiany kontekstu, więc nie wprowadza wielowątkowości znanej z pełnoprawnych systemów operacyjnych. Jego celem jest zakolejkowanie zadań, oznaczenie ich jako gotowych do wykonania po upływie wymaganego czasu, a następnie wykonaniu ich przy pierwszej sposobności po wyjściu programu z przerwania. Kod służący do  kolejkowania zadań przedstawiono na listingu \ref{listing_scheduler_add}. Na listingu \ref{listing_scheduler_check} umieszczono funkcję, która jest cyklicznie wywyływana (co 10 ms) w przerwaniu generowanym przez energooszczędny timer. Jej zadaniem jest oznaczenie zadań, które można już wykonać. Na ostatnim listingu, oznaczonym numerem  \ref{listing_scheduler_exec}, przedstawiono procedurę wykonującą oczekujące zadania. Ograniczeniem funkcjonalnym jest tutaj fakt, iż żadne z zadań nie może przyjmować argumentów, ani zwracać wartości. 
\clearpage

\begin{lstlisting}[label=listing_scheduler_add, caption=Funkcja do kolejkowania zadań]

scheduler_error_code_e SchedulerAddOperation(void (*callback)(void),
	 										volatile	uint32_t timeMsFromNow,
	 										volatile uint8_t* taskIndex,
	 										bool isCyclic){
    scheduler_entry_t entry;

    // Just safe guard not to miss the time
    if (timeMsFromNow < 2)
    {
        timeMsFromNow = 2;
    }

    for (uint8_t i=0; i< SCHEDULER_BUFFER_SIZE; ++i)
    {
        if (_scheduleBuffer[i].isInProgress == false)
        {
            entry.isInProgress = true;
            entry.isTimedOut = false;
            entry.callback = callback;
            entry.timePeriodMs = timeMsFromNow;
            entry.triggerTime = scheduler_current_time_ms + timeMsFromNow;
            entry.isCyclic = isCyclic;
            memcpy(&_scheduleBuffer[i], &entry, sizeof(scheduler_entry_t));
            if (taskIndex != NULL)
                *taskIndex = i;
            return E_SCHEDULER_OK;
        }
    }

    return E_SCHEDULER_NO_RESOURCES;
}
\end{lstlisting}

\clearpage
\begin{lstlisting}[label=listing_scheduler_check, caption=Funkcja do sprawdzania czy nie należy wykonań zadania]

scheduler_error_code_e SchedulerCheckOperations(){
    scheduler_current_time_ms += 10;
    for (uint8_t i=0; i< SCHEDULER_BUFFER_SIZE; ++i)
    {
        if (_scheduleBuffer[i].isInProgress == true &&
            _scheduleBuffer[i].triggerTime <= scheduler_current_time_ms)
        {
            // If it is cyclic task - reschedule the next cycle
            if (_scheduleBuffer[i].isCyclic)
            {
                _scheduleBuffer[i].triggerTime = scheduler_current_time_ms 
                +_scheduleBuffer[i].timePeriodMs;
            }

            _scheduleBuffer[i].isTimedOut = true;
        }
    }

    return E_SCHEDULER_OK;
}
\end{lstlisting}


\begin{lstlisting}[label=listing_scheduler_exec, caption=Funkcja do wykonywania zadań]

scheduler_error_code_e ScheduleExecutePendingOperations(){
    for (uint8_t i=0; i< SCHEDULER_BUFFER_SIZE; ++i)
    {
        if (_scheduleBuffer[i].isInProgress == true && 
        		_scheduleBuffer[i].isTimedOut == true)
        {
            if (_scheduleBuffer[i].isCyclic == false)
            {
                _scheduleBuffer[i].isInProgress = false;
            }
            _scheduleBuffer[i].callback();
            _scheduleBuffer[i].isTimedOut = false;
        }
    }

    return E_SCHEDULER_OK;
}

\end{lstlisting}

Główny cykl działania urządzenia został przedstawiony na rysunku \ref{fig:image_soft_mainboard_main_alghoritm}.

\begin{figure}[H]
	\centering
	\includegraphics[width=16cm]{img/software/mainboard/Tracking_alghoritm.jpg}
	\caption{Główny algorytm działania urządzenia. 
	\\Źródło: Opracowanie własne.}
	\label{fig:image_soft_mainboard_main_alghoritm}
\end{figure}

Jak widać na rysunku \ref{fig:image_soft_mainboard_main_alghoritm}, po uruchomieniu i zainicjalizowaniu peryferiów i modułów na płytce, mikrokontroler dokonuje sprawdzenia czy wygenerowany został klucz szyfrujący, służący do zabezpieczenia komunikacji z dedykowanym urządzeniem deaktywującym. Jeśli klucza nie ma, to oznacza że urządzenie nie zostało jeszcze sparowane. Wówczas, włączony zostaje układ NFC, procesor wchodzi w tryb oszczędzania energii w trakcie oczekiwania na parowanie, a całe urządzenie przechodzi w stan nieoperacyjny, dopóki nie zostanie powiązane z modułem deaktywującym. Algorytm komunikacji został przedstawiony w podrozdziale \ref{key_tag} na rysunku \ref{fig:image_soft_keytag_key exchange}. Po deaktywacji, moduł NFC zostaje wyłączony i nie jest używany aż do momentu powrotu do ustawień fabrycznych, a mikrokontroler wchodzi w tryb oszczędzania energii w oczekiwaniu na nadchodzące zadania.

Pierwszym z nich jest wykrywanie ruchu pojazdu. Zostało to zrealizowane poprzez wykorzystanie funkcji akcelerometru - wybudzenia w razie wykrycia przyspieszenia powyżej programowalnego progu, które utrzymywałoby się przez pewien konfigurowalny czas. W wyniku badań eksperymentalnych, został on ustalony na wartość 0,9 $\frac{m}{s^2}$, trwającą przez więcej niż 1 sekundę, co pozwala na wykrycie drgań spowodowanych zamknięciem drzwi pojazdu. Akcelerometr jest cykliczne (co 5 sekund) odpytywany przez mikrokontroler, w celu sprawdzenia czy nie nastąpił ruch pojazdu. Jeśli nie zostało to wykryte, to mikrokontroler przechodzi do trybu oszczędzania energii i cały cykl się powtarza. Jeśli wykryto ruch, zadanie okresowego sprawdzania przemieszczenia jest wyłączane, natomiast do kolejki zadań ładowane są 3 najważniejsze z punktu widzenia całego systemu procedury - zadanie alarmu, skanowania w poszukiwaniu \textit{Key Tag'a} oraz cyklicznego pobierania próbek lokalizacji. Pierwsze z nich stanowi zegar, który odlicza 30 sekund. Jeśli w tym czasie, alarm zostanie deaktywowany poprzez nawiązanie połączenia z urządzeniem dezaktywującym i nadanie odpowiedniego komunikatu, wówczas zadanie zostaje anulowane i jedynym cyklicznym zadaniem jest próbkowanie lokalizacji. W przeciwnym razie, uruchamiane jest zadanie cyklicznego, co 10 minutowego powiadamiania właściciela o lokalizacji pojazdu poprzez wiadomości SMS. Okres ten został dobrany w ten sposób, aby nie wyczerpać za szybko śródków na koncie karty SIM, użytej w module, a przy tym uzyskać rozsądną częstotliwość wysyłania SMS'ów. Zadanie to można anulować, w przypadku fałszywego alarmu, wysyłając odpowiednią komendę poprzez SMS z numeru właściciela. 

Kolejne zadanie to alarm w Pierwszy krok w procesie deaktywacji alarmu jest wykonywany przez płytkę główną. Ze względu na fakt, iż urządzenie powinno być ukryte, nie powinno ono rozgłaszać żadnych pakietów w sposób ciągły, co zminimalizuje jego wykrywalność w aplikacjach skanujących poprzez Bluetooth Low Energy. Z tego powodu, w momencie wykrycia ruchu pojazdu, to ono nawiązuje połączenie z \textit{Key Tag'iem}. W tym celu dokonuje skanowania urządzeń posiadających odpowiedni zestaw serwisów i charakterystyk. Dla każdego z nich, sprawdza jego nazwę. Jeśli urządzenie ma odpowiednią nazwę, dopiero wówczas nawiązywane jest z nim połączenie. W trakcie połączenia, urządzenie peryferyjne musi przesłać kod deaktywujący (wygenerowany w trakcie parowania), zaszyfrowany kluczem szyfrowania wygenerowanym na potrzeby danego połączenia. Jeśli zostanie on poprawnie odszyfrowany, alarm jest deaktywowany. W przeciwnym razie - połączenie zostaje zerwane. Algorytm deaktywowania alarmu przedstawiono w rozdziale \ref{key_tag} na rysunku \ref{fig:image_soft_keytag_alarm_deactivation}.

Ponadto, niezależnie od tego czy alarm został deaktywowany czy nie, uruchamiane jest zadanie cyklicznego pobierania próbek lokalizacji. Okres próbkowania wynosi 10 sekund i w momencie pobierania informacji gromadzone są dane takie jak:

\begin{itemize}
\item status lokalizacji,
\item lokalizacja pojazdu,
\item prędkość pojazdu,
\item średnie przyspieszenie pojazdu z okresu pomiędzy próbkami,
\item azymut ruchu,
\item ocena jazdy z okresu pomiędzy próbkami,
\item parametr HDOP informujący o jakości sygnału GPS,
\item liczba satelitów z których odebrano sygnał,
\item czas pobrania próbki.
\end{itemize}

Na rysunku \ref{fig:image_soft_mainboard_control_flow} przedstawiono diagram interakcji, który pokazuje przepływ sterowania pomiędzy zadaniami. 


\begin{figure}[H]
	\centering
	\includegraphics[width=17cm]{img/software/mainboard/MainBoardStartTracking.jpg}
	\caption{Przepływ sterowania w przypadku wykrycia ruchu pojazdu. 
	\\Źródło: Opracowanie własne.}
	\label{fig:image_soft_mainboard_control_flow}
\end{figure}