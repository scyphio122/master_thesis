\section{Urządzenie deaktywujące}
\label{key_tag}

Urządzenie deaktywujące stanowi najprostszy moduł w całym systemie. W jego obrębie nosi miano \textit{Key Tag'a}, ze względu na założenie, iż będzie się ono znajdować przy kluczach pojazdu. Z racji tego, że posiada on tylko jedną, ale jakże istotną funkcję - deaktywację alarmu, jego główną cechą powinna być energooszczędność. Z tego powodu, urządzenie to  pozbawione jest zewnętrznych układów poza anteną BLE i przez większość czasu znajduje się w trybie oszczędzania energii. 

Aby system w samochodzie stał się operacyjny, pierwszym krokiem jest sparowanie ze sobą modułu lokalizującego z przeznaczonym dla niego urządzeniem deaktywacyjnym. Proces ten polega na wygenerowaniu przez urządzenie lokalizujące 16-bajtowego klucza szyfrującego dla algorytmu AES128, oraz 16-bajtowej komendy deaktywującej po stronie \textit{Key Tag'a}. Losowanie komendy deaktywującej zamiast wprowadzenie jej jako stałej do pamięci mikrokontrolera wszystkich urządzeń stanowi dodatkowe zabezpieczenie systemu. Jak zostało przedstawione we wcześniejszych rozdziałach, operacja parowania realizowana jest poprzez interfejs NFC. Przedstawiono ją na rysunku \ref{fig:image_soft_keytag_key exchange}.

\begin{figure}[H]
	\centering
	\includegraphics[width=8.5cm]{img/software/keytag/Key_exchange.jpg}
	\caption{Przepływ sterowania w trakcie parowania urządzenia lokalizującego z urządzeniem deaktywującym. 
	\\Źródło: Opracowanie własne.}
	\label{fig:image_soft_keytag_key exchange}
\end{figure}

\pagebreak
W momencie wykrycia ruchu pojazdu, urządzenie lokalizujące uaktywnia mechanizm skanowania urządzeń wykorzystujących BLE, w poszukiwaniu \textit{Key Tag'a}. Gdy znajdzie urządzenie o pasującej specyfikacji (nazwa oraz struktura serwisów i charakterystyk), łączy się z nim i generuje tymczasowy 16-bajtowy klucz szyfrujący na potrzeby aktualnego połączenia, który jest następnie szyfrowany kluczem głównym i wysyłany do urządzenia deaktywującego. W kolejnym kroku \textit{Key Tag} dokonuje deszyfrowania klucza tymczasowego, i zaszyfrowania nim kodu deaktywyującego, wylosowanego na etapie parowania. Jest on przesyłany z powrotem do urządzenia lokalizującego, które sprawdza sprawdza czy klucz jest poprawny i jeśli tak - deaktywuje alarm. Jeśli nie, alarm pozostaje aktualny. W każdym wypadku, po przesłaniu klucza, połączenie zostaje przerwane, a urządzenie lokalizujące wyłącza skanowanie. Zostanie ono włączone dopiero gdy zostanie wykryty nowy ruch pojazdu. Schemat operacji deaktywacji przedstawiono na rysunku \ref{fig:image_soft_keytag_alarm_deactivation}.

Dzięki takiemu rozwiązaniu, następuje minimalizacja liczby nawiązywanych połączeń. Jest to korzystne z punktu widzenia obu urządzeń, ponieważ połączenie poprzez Bluetooth Low Energy stanowi najbardziej energochłonny element komunikacyjny pomiędzy urządzeniami.  

\begin{figure}[H]
	\centering
	\includegraphics[width=17cm]{img/software/keytag/alarm_deactivation.jpg}
	\caption{Przepływ sterowania w momencie deaktywacji alarmu. 
	\\Źródło: Opracowanie własne.}
	\label{fig:image_soft_keytag_alarm_deactivation}
\end{figure}