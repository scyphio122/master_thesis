\chapter{Podsumowanie}
\label{summary}

Celem pracy było zaprojektowanie, wykonanie i oprogramowanie urządzenia stanowiącego dodatkowe zabezpieczenie antykradzieżowe pojazdu, w postaci lokalizatora wykorzystującego system GNSS oraz GSM, zdolnego do analizy stylu jazdy kierowcy, a także całego systemu informatycznego, który pozwoliłby na obsłużenie pozyskanych danych.

W jej ramach zaprojektowano oraz wykonano dwa urządzenia - lokalizujące, stanowiące główny ciężar funkcjonalny pracy oraz pomocnicze, służące do deaktywacji alarmu, czyli autoryzacji ruchu pojazdu. Urządzenia udało się poprawnie uruchomić oraz zaprogramować w celu wykonywania przeznaczonych im zadań. Komunikację pomiędzy nimi zabezpieczono poprzez zastosowanie szyfru AES128. Dodatkowy wzrost bezpieczeństwa został osiągnięty dzięki implementacji algorytmu zmiennego klucza szyfrującego.

Urządzenie lokalizujące poprawnie zbiera dane o lokalizacji, prędkości i kursie (azymucie) poprzez wykorzystanie systemu GPS oraz o przyspieszeniu dzięki zastosowanemu w nim akcelerometrowi. W celu przetestowania tej funkcjonalności wykonano szereg testów drogowych, które miały na celu potwierdzenie przydatności urządzenia oraz dokładności lokalizacji. Wszystkie wypadły pozytywnie, pozycjonując pojazd na mapie z dokładnością do 2 - 3 metrów. Co więcej, urządzenie z sukcesem wysyła dane wykorzystując protokół HTTP, a także odbiera i transmituje krótkie wiadomości SMS oraz aktualizuje swój wewnętrzny zegar na podstawie czasu pobranego z sieci GSM.

W celu gromadzenia danych niezbędnych do spełnienia założonych celów, napisano aplikację serwerową w języku C++, która obsługuje bazę danych SQL oraz zapytania HTTP. Służą one zarówno do odbierania danych z urządzenia lokalizującego jak i komunikacji ze stroną internetową napisaną w języku Java Script. Jej zadaniem jest wizualna reprezentacja tras przebytych przez pojazdy, a także powiązanych z nimi danych opisujących ich przebieg upoważnionym do tego użytkownikom. Oba te elementy zostały umieszczone na domowym serwerze w postaci minikomputera Raspberry PI i są dostępne z internetu. 

Zwieńczeniem prac były eksperymentalne badania oraz zaproponowanie algorytmu służącego do oceny stylu jazdy kierowców. W trakcie badań wykonano szereg pomiarów, mających na celu ustalenie zależności między przyspieszeniem i jego pochodną - zrywem oraz częstotliwości ich próbkowania, a sposobem jazdy kierowców. Na ich podstawie z sukcesem zaimplementowano w urządzeniu tę funkcjonalność. Wykonane samodzielnie testy drogowe potwierdziły skuteczność proponowanej metody.

Pomimo spełnienia wszystkich założeń i celów, należy przedstawić pewne możliwości dalszego rozwoju projektu. Aby opisany w pracy system stał się pełnowartościowym produktem, który mógłby trafić do konsumentów, należy wprowadzić kilka dodatkowych elementów. Są to:

\begin{enumerate}
\item Aplikacja mobilna na systemy Android i IOS - w dzisiejszych czasach, smartfony zajmują równoważne miejsce w stosunku do komputerów. Są one dodatkowym łącznikiem ze światem technologii, który wydaje się być idealny wręcz do konfiguracji urządzenia, komunikacji z nim i reprezentacji danych.
\item Bootloader - jest to niezbędny fragment programu urządzeń wbudowanych, który pozwala na zdalną aktualizację oprogramowania. O sile produktu, oprócz jego możliwości, stanowi bowiem jego zdolność do szybkiego rozwoju i odpowiedzi na nieprzewidziane wcześniej wymagania i sytuacje.
\item Wykorzystanie biblioteki map drogowych - dzięki temu można jeszcze dokładniej przypisać próbki do dróg i adresów, a także powiązać je z dodatkowymi danymi jak na przykład ograniczenia prędkości obowiązujących na przebytych drogach.
\item Umożliwienia komunikacji alarmowej poprzez wiadomości e-mail - pozwoliłoby to na zmniejszenie kosztów ponoszonych w razie alarmu, ponieważ koszt wysłania e-mail'a jest nieporównanie niższy od kosztu wiadomości SMS.
\item Wykorzystanie systemów GLONASS i Beidou - dzięki temu uzyskano by jeszcze większą dokładność lokalizacji poprzez GNSS.
\item Zaprojektowanie wariantu urządzenia wykorzystującego sieć LTE-M bądź LORA - stosunkowo nowe standardy komunikacji bezprzewodowej. Pierwszy z nich umożliwia zwiększenie przepustowości transmisji, a drugi - jej energooszczędności.
\item Wprowadzenie do algorytmu analizy stylu jazdy zdolności "uczenia", czyli funkcjonalności samoczynnej aktualizacji zaprogramowanych progów określających czułość oceny.
\end{enumerate} 

Podsumowując, uważam opisywany w niniejszej pracy projekt za zakończony w pełni sukcesem. Zrealizowano wszystkie założone cele, pokonano wszystkie napotkane problemy, zgromadzone dane posiadają dużą dokładność, a tym samym potencjalną wartość rynkową. Mocną stroną systemu są jego ograniczone wymiary, możliwość podglądu lokalizacji pojazdów na bieżąco oraz całkowita niezależność od instalacji elektrycznej pojazdu. Dzięki zastosowaniu funkcji oceny stylu jazdy, a także alarmu w przypadku kradzieży stanowi produkt przewyższający dostępną na rynku konkurencję.