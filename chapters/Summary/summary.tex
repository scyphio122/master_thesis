\chapter{Podsumowanie}
\label{summary}

Celem pracy było zaprojektowanie i wykonanie urządzenia do lokalizacji pojazdów w trybie on- if offline, zapewniającego możliwość analizy stylu jazdy kierowcy oraz całego systemu informatycznego pozwalającego na gromadzenie danych. W jej ramach powstały dwa urządzenia - lokalizujące, stanowiące główny ciężar funkcjonalny pracy oraz pomocnicze, służące do deaktywacji alarmu, czyli autoryzacji ruchu pojazdu. Ponadto, napisano aplikację serwerową, która obsługuje bazę danych SQL oraz zapytania HTTP, poprzez które możliwa jest komunikacja z nią. Ostatnim elementem jest strona internetowa napisana w języku HTTP, przy użyciu języka JavaScript umożliwiającego interakcję. Zgodnie z założeniami pracy, komunikacja pomiędzy urządzeniami jest szyfrowana szyfrem zmiennym, zwiększającym dodatkowo bezpieczeństwo. Ponadto udało się zaimplementować prosty, lecz skuteczny algorytm analizy stylu jazdy kierowcy, który jest odpowiedzią na potrzeby rynkowe. 
Mimo osiągnięcia wszystkich założonych celów, w pracy wciąż jest miejsce na ulepszenia. Aby opisany w niej sysstem stał się pełnowartościowym produktem, który mógłby trafić do konsumentów, należy wprowadzić kilka dodatkowych elementów. Są to:

\begin{enumerate}
\item Aplikacja mobilna na systemy Android i IOS - w dzisiejszych czasach, smartfony zajmują równoważne miejsce w stosunku do komputerów. Są one dodatkowym łącznikiem ze światem technologii, który wydaje się być idealny wręcz do konfiguracji urządzenia, komunikacji z nim i przedstawiania danych.
\item Bootloader - jest to niezbędny fragment programu urządzeń wbudowanych, który pozwala na zdalną aktualizację oprogramowania. O sile produktu, oprócz jego możliwości, stanowi bowiem jego zdolność do szybkiego rozwoju i odpowiedzi na nieprzewidziane wcześniej wymagania i sytuacje.
\item Wykorzystanie biblioteki map drogowych - dzięki temu można jeszcze dokładniej przypisać próbki do dróg i adresów, a także powiązać je z dodatkowymi danymi jak na przykład ograniczenia prędkości obowiązujących na przebytych drogach.
\item Umożliwienia komunikacji alarmowej poprze wiadomości e-mail - pozwoliłoby to na zmniejszenie kosztów ponoszonych w razie alarmu, ponieważ koszt wysłania e-mail'a jest nieporównanie niższy od kosztu wiadomości SMS.
\item Wykorzystanie systemów GLONASS i Beidou - dzięki temu uzyskano by jeszcze większą dokładność lokalizacji poprzez GNSS.
\item Zaprojektowanie wariantu urządzenia wykorzystującego sieć LTE-M bądź LORA - są to stosunkowo nowe standardy komunikacji bezprzewodowej. Pierwszy z nich umożliwia zwiększenie przepustowości transmisji, a drugi - jej energooszczędności.
\end{enumerate} 

Podsumowując, uważam opisywany w niniejszej pracy projekt za zakończony w pełni sukcesywnie. Zrealizowano wszystkie założone cele oraz pokonano wszystkie napotkane problemy. 