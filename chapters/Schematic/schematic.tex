\chapter{Schematy elektroniczne urządzeń}
\label{schematics}

\section{Urządzenie lokalizujące}
Ze względu na poziom skomplikowania układu, schemat elektroniczny został przedstawiony w postaci serii podschematów. W urządzeniu lokalizującym można wyróżnić trzy podstawowe moduły elektroniczne. Są to:

\begin{itemize}
\item Moduł zasilania, przedstawiony na rysunku \ref{fig:image_mainboard_power_schematic}
\item Moduł funkcjonalny, przedstawiony na rysunku \ref{fig:image_mainboard_functional_schematic}
\item Moduł NFC, przedstawiony na rysunku \ref{fig:image_mainboard_NFC_schematic}
\end{itemize}

\begin{figure}[H]
	\centering
	\includegraphics[width=16cm, height=23cm]{img/schematics/mainboard_power.jpg}
	\caption{Schemat modułu zasilania urządzenia lokalizującego. \\ Źródło: Opracowanie własne.}
	\label{fig:image_mainboard_power_schematic}
\end{figure}

\begin{figure}[H]
	\centering
	\includegraphics[width=16cm, height=23cm]{img/schematics/mainboard_fixed.png}
	\caption{Schemat modułu funkcjonalnego urządzenia lokalizującego. \\ Źródło: Opracowanie własne.}
	\label{fig:image_mainboard_functional_schematic}
\end{figure}

\begin{figure}[H]
	\centering
	\includegraphics[width=16cm, height=23cm]{img/schematics/mainboard_NFC.jpg}
	\caption{Schemat modułu NFC urządzenia lokalizującego. \\ Źródło: Opracowanie własne.}
	\label{fig:image_mainboard_NFC_schematic}
\end{figure}



\subsection{Układ zasilania}

Akumulator samochodowy jest bardzo wygodnym źródłem zasilania układów elektronicznych. Bliska odległość do alternatora i innych urządzeń o obciążeniu indukcyjnym powoduje generowanie silnych zakłóceń na linii zasilającej. Niekiedy "szpilki" napięciowe osiągają wartość rzędu 100 V. Z tego powodu należy stosować odpowiednie zabezpieczenia (diody zabezpieczające). Powodują one ograniczenie  napięcia do pewnej bezpiecznej wartości. W konstrukcji zastosowane ograniczenie napięciowe wynosi 24,4 V. Zabezpieczeniem przeciążeniowym nadprądowym jest bezpiecznik samochodowy o wartości 4 A. 
Ponieważ jednym z wymagań układu jest możliwość zasilania bateryjnego, konieczne jest zastosowanie dodatkowego przyłącza zasilania. Urządzenie można zasilić dowolną baterią o napięciu od 4 V do 24 V i wydajności prądowej co najmniej 3 A w szczycie. Ze względu na prawdopodobieństwo wystąpienia różnic napięć pomiędzy dodatkową baterią, a akumulatorem samochodu i wiążącym się z tym przepływem prądu z jednego źródła do drugiego, konieczne jest zastosowanie diód zabezpieczających przed rozładowaniem baterii przez akumulator (gdy napięcie akumulatora jest niższe niż napięcie baterii) lub mogącym doprowadzić baterię do zniszczenia doładowywaniem jej bezpośrednio z akumulatora (gdy napięcie baterii jest niższe od napięcia akumulatora). W trakcie projektowania, zdecydowano się na zastosowanie diód Schottky’ego ze względu na niski spadek napięcia w kierunku przewodzenia (0,2 V - 0,55 V, w zależności od natężenia prądu) oraz szybki czas przełączania ze stanu zaporowego do przewodzenia (ograniczenie krótkotrwałych zaników zasilania przy wyłączaniu pojazdu). Na rysunku \ref{fig:image_mainboard_power_input} przedstawiono układ wejściowy zasilania urządzenia lokalizującego.

\begin{figure}[H]
	\centering
	\includegraphics[width=10cm]{img/schematics/mainboard_power_input.jpg}
	\caption{Schemat modułu zasilania wejściowego urządzenia lokalizującego. \\ Źródło: Opracowanie własne.}
	\label{fig:image_mainboard_power_input}
\end{figure}

Ponieważ napięcie wejściowe jest zbyt wysokie do zasilenia układu lokalizatora, stało się koniecznym zastosowanie przetwornicy DC/DC. Schemat wykorzystanej przetwornicy przedstawiono na rysunku \ref{fig:image_mainboard_power_converter}.

\begin{figure}[H]
	\centering
	\includegraphics[width=17cm, height=8cm]{img/schematics/mainboard_power_converter.jpg}
	\caption{Schemat przetwornicy impulsowej modułu zasilania urządzenia lokalizującego. \\ Źródło: Opracowanie własne.}
	\label{fig:image_mainboard_power_converter}
\end{figure}

Zastosowana w urządzeniu przetwornica umożliwia zasilanie napięciami od 4 V do 38 V. Wybrano ją ze względu na niewielką liczbę, w porównaniu do innych modułów, zewnętrznych komponentów, niezbędnych do jej działania, a także wysoką sprawność rzędu od 85\% do 90\% w zależności od chwilowego natężenia prądu. Wytwarza ona na wyjściu napięcie o wartości 4 V, którym zasilany jest moduł GSM oraz dalszy stopień obniżania napięcia.

Dodatkowo, z napięcia wyjściowego z przetwornicy uzyskiwane jest napięcie o wartości 3,3 V. Jest ono niezbędne do zasilania układów mikrokontrolera, pamięci flash, akcelerometru oraz układu GPS. Szacowany maksymalny pobór prądu przez te układy wynosi ok. 200 mA, stąd uwzględniając odpowiedni zapas zastosowano stabilizator napięcia LDO (\textit{ang. \textbf{L}ow \textbf{D}ropout \textbf{S}tabilizer}) o maksymalnym natężeniu prądu wyjściowego 0,5 A. Jego schemat przdstawiono na rysunku \ref{fig:image_mainboard_power_ldo}.

\begin{figure}[H]
	\centering
	\includegraphics[width=12cm]{img/schematics/mainboard_power_ldo.jpg}
	\caption{Schemat stabilizatora napięcia modułu zasilania urządzenia lokalizującego. \\ Źródło: Opracowanie własne.}
	\label{fig:image_mainboard_power_ldo}
\end{figure}

\subsection{Moduł mikrokontrolera}

Układ lokalizatora wykorzystuje mikrokontroler nRF52832 firmy Nordic Semiconductor. Układ ten posiada 32 bitowy rdzeń Cortex-M4 zaprojektowany przez firmę ARM, sprzętową jednostkę FPU, 512 KB wewnętrznej pamięci Flash oraz 64 KB pamięci RAM. Zdecydowano się na wykorzystanie tego mikrokontrolera ze względu na kilka czynników. Pierwszym z nich jest jego architektura - posiada wbudowany układ radiowy działający na częstotliwości 2,4 GHz i umożliwiający komunikację w standardzie Bluetooth Low Energy, ANT lub wykorzystanie własnego protokołu. Dodatkowym atutem tego mikrokontrolera jest wyposażenie go w sprzętowy interfejs NFCT, umożliwiający wykorzystanie modułu jako tag (urządzenie podrzędne) w komunikacji poprzez interfejs NFC. Ponadto ma bardzo duże możliwości obliczeniowe – wbudowany wewnętrzny zegar taktujący o częstotliwości 64 MHz umożliwia bardzo szybkie wykonywanie zaprogramowanych zadań i szybki powrót do trybu oszczędzania energii. Zużycie energii przez ten procesor jest bardzo niewielkie. W trakcie wykonywania programu pobór prądu wynosi 58 $\mu A$ /MHz gdy kod wykonywany jest z pamięci flash, natomiast w trybie oszczędzania energii pobór spada do ok 1,9 $\mu A$. Ostatnim i być może najważniejszym czynnikiem decydującym na wybranie tego układu jest posiadane przez autora doświadczenie zawodowe w programowaniu układów od tego producenta, a zatem bardzo dobra znajomość jego możliwości i SDK (\textit{ang. \textbf{S}oftware \textbf{D}evelopment \textbf{K}it}). Schemat mikrokontrolera przedstawiono na rysunku \ref{fig:image_mainboard_functional_mcu}.

\begin{figure}[H]
	\centering
	\includegraphics[width=15cm]{img/schematics/mainboard_functional_mcu.jpg}
	\caption{Schemat modułu mikrokontrolera w urządzeniu lokalizującym. \\ Źródło: Opracowanie własne.}
	\label{fig:image_mainboard_functional_mcu}
\end{figure}

\subsection{Moduł GSM i GPS}

Jako moduł realizujący funkcję lokalizacji i głównej transmisji w urządzeniu wybrano układ Quectel MC60. Stanowi on połączenie modułu GSM oraz GPS w jednym chipie. Umożliwia wykorzystanie wielu protokołów, takich jak: TCP/IP, UDP, FTP, PPP, HTTP czy NTP. Ponadto możliwy jest odbiór i nadawanie danych w postaci krótkich wiadomości SMS. Układ posiada niewielkie wymiary: 18,7 mm x 16 mm x 2,1 mm, dzięki czemu możliwe jest zmniejszenie rozmiarów całego urządzenia. Zużycie energii wynosi:
\begin{itemize}
\item Około 25 mA gdy działa jedynie moduł GPS
\item Do 1,5 A w trakcie transmisji danych poprzez sieć GSM
\end{itemize}

Ponadto, kombinacja tych dwóch systemów umożliwia wykorzystanie funkcjonalności AGPS. Polega ona na podaniu do modułu GPS zgrubnych danych o położeniu satelitów, pobranych z sieci GSM. Dzięki temu, ustalenie własnej lokalizacji, nawet po długotrwałym braku zasilania, trwa ok. sekundy (tzw. \textit{warm start}). Schemat modułu GSM i GPS przedstawiono na rysunku \ref{fig:image_mainboard_functional_gps_gsm}.

\begin{figure}[H]
	\centering
	\includegraphics[width=15cm]{img/schematics/mainboard_gps_fixed.png}
	\caption{Schemat modułu układu GSM i GPS w urządzeniu lokalizującym. \\ Źródło: Opracowanie własne.}
	\label{fig:image_mainboard_functional_gps_gsm}
\end{figure}

W celu zwiększenia niezawodności działania urządzenia, zdecydowano zastosować zewnętrzne anteny GSM i GPS, poprawiające jakość sygnału. Dodatkowo, antena GPS jest anteną aktywną. Oznacza to, że dostarczane jest do niej dodatkowe zasilanie, powodujące wzmocnienie odebranego sygnału. Schemat anten przedstawiono na rysunku \ref{fig:image_mainboard_functional_gps_gsm_antennas}. Zawarte na nim znaki zapytania zamiast wartości pojemności kondensatorów oznaczają, że należy je dobrać po zmontowaniu układu i przebadaniu jej pod kątem jak najlepszego dopasowania impedancji. Na etapie uruchomienia układu okazało się, iż kondensatory te nie są niezbędne w celu poprawnego działania urządzenia.

\begin{figure}[H]
	\centering
	\includegraphics[width=15cm]{img/schematics/mainboard_functional_gps_gsm_antennas.jpg}
	\caption{Schemat modułu anten dla GSM i GPS w urządzeniu lokalizującym. \\ Źródło: Opracowanie własne.}
	\label{fig:image_mainboard_functional_gps_gsm_antennas}
\end{figure}

Układ GSM wymaga połączenia z kartą SIM, umożliwiającą zalogowanie do sieci. Przedstawiono je na rysunku \ref{fig:image_mainboard_functional_gsm_sim_card}. Widać na nim układ TVS, który jest odpowiedzialny za zabezpieczenie karty SIM przed wyładowaniami elektrostatycznymi ESD (\textit{ang. \textbf{E}lectro\textbf{s}tatic \textbf{d}ischarge}).

\begin{figure}[H]
	\centering
	\includegraphics[width=15cm]{img/schematics/mainboard_gsm_sim_card.jpg}
	\caption{Schemat modułu karty SIM w urządzeniu lokalizującym. \\ Źródło: Opracowanie własne.}
	\label{fig:image_mainboard_functional_gsm_sim_card}
\end{figure}

\subsection{Moduł pamięci flash}

Wewnętrzna pamięć flash mikrokontrolera jest niewystarczająca, aby przechowywać w niej przebyte trasy wraz z parametrami jazdy. Stąd też pojawia się konieczność zastosowania zewnętrznego układu pamięci nieulotnej. Zastosowana w urządzeniu pamięć flash posiada pojemność 8 MB, co umożliwi przechowywanie wielu długich tras wraz z dodatkowymi parametrami je opisującymi. Schemat podłączenia pamięci w urządzeniu lokalizującym pokazano na rysunku \ref{fig:image_mainboard_functional_flash}.

\begin{figure}[H]
	\centering
	\includegraphics[width=15cm]{img/schematics/mainboard_functional_flash_memory.jpg}
	\caption{Schemat modułu pamięci flash w urządzeniu lokalizującym. \\ Źródło: Opracowanie własne.}
	\label{fig:image_mainboard_functional_flash}
\end{figure}

\subsection{Moduł akcelerometru}

Kolejną ważną częścią urządzenia jest moduł akcelerometru. Pozwala on na wygenerowanie przerwania i wybudzenie urządzenia w momencie wykrycia ruchu pojazdu, a w razie braku deaktywacji funkcji - uruchomienie procedury alarmowej. Ponadto, dzięki jego wskazaniom możliwe jest wyznaczenie przyspieszenia pojazdu, pozwalające na ocenę i profilowanie stylu prowadzenia pojazdu przez kierowcę. Wbudowany żyroskop pozwala na dokładniejsze profilowanie stylu jazdy kierowcy w trakcie pokonywania zakrętów oraz zmiany pasa.
Schemat modułu akcelerometru przedstawiono na rysunku \ref{fig:image_mainboard_functional_accelerometer}.

\begin{figure}[H]
	\centering
	\includegraphics[width=15cm]{img/schematics/mainboard_functional_accelerometer.jpg}
	\caption{Schemat modułu akcelerometru w urządzeniu lokalizującym. \\ Źródło: Opracowanie własne.}
	\label{fig:image_mainboard_functional_accelerometer}
\end{figure}

\subsection{Moduł NFC}

Moduł ten stanowi istotną część z punktu widzenia bezpieczeństwa komunikacji bezprzewodowej. Jest ono zapewnione poprzez zastosowanie szyfrowania wiadomości. Jeśli jednak ktoś podsłucha transmisję inicjalizacji urządzenia, w której przekazywane są klucze szyfrujące, całe zabezpieczenie traci sens. Dzięki zastosowaniu modułu NFC, możliwość podsłuchania transmisji wymiany kluczy szyfrujących zostaje zniwelowana poprzez fizyczne ograniczenia zasięgu komunikacji. NFC posiada bowiem maksymalny zasięg do 10 cm.
Komunikacja odbywa się pomiędzy dwoma urządzeniami. Ze względu na sposób transmisji, jedno z urządzeń inicjuje komunikację. Inicjator generuje zmienne pole magnetyczne, w którym może (lecz nie musi) kodować dane wysyłane do urządzenia docelowego. Urządzenie docelowe wykrywa to pole i może odpowiedzieć poprzez odpowiednie zniekształcenie go, które jest wykrywane przez inicjator. Urządzenie docelowe nie generuje żadnego pola magnetycznego. Może jedynie zniekształcać pole generowane przez inicjator. Stąd wynika, że inicjator musi mieć znacznie większe zużycie energii niż urządzenie docelowe – tag. W urządzeniu lokalizacyjnym zastosowano moduł inicjatora NFC, którego schemat przedstawiono na rysunkach \ref{fig:image_mainboard_NFC_digital} - część cyfrowa oraz \ref{fig:image_mainboard_NFC_analog} - część analogowa.

\begin{figure}[H]
	\centering
	\includegraphics[width=15cm]{img/schematics/mainboard_NFC_chip.jpg}
	\caption{Schemat części cyfrowej modułu NFC w urządzeniu lokalizującym. \\ Źródło: Opracowanie własne.}
	\label{fig:image_mainboard_NFC_digital}
\end{figure}

\begin{figure}[H]
	\centering
	\includegraphics[width=15cm]{img/schematics/mainboard_NFC_analog.jpg}
	\caption{Schemat części analogowej modułu NFC w urządzeniu lokalizującym. \\ Źródło: Opracowanie własne.}
	\label{fig:image_mainboard_NFC_analog}
\end{figure}


\clearpage
\section{Urządzenie deaktywujące}

Głównym zadaniem tego urządzenia jest cykliczne rozgłaszanie. Po wykryciu przez urządzenie lokalizujące, łączy się ono z deaktywatorem i bezpiecznym kanałem dokonywane jest wyłączenie funkcji alarmu. Dzięki elementarnej funkcji, którą wykonuje, możliwe jest zasilenie go ze standardowej baterii CR2032 o promieniu 20 mm i grubości 3,2 mm. Urządzenie to, przy odpowiedniej konfiguracji parametrów transmisji może działać kilka lat bez konieczności jej wymiany. Zastosowanie wspomnianego źródła zasilania stanowi kompromis pomiędzy czasem działania i rozmiarem urządzenia, które docelowo powinno być umieszczone przy kluczach samochodowych. Schemat deaktywatora przedstawiono na rysunku \ref{fig:image_mainboard_NFC_schematic}.

\begin{figure}[H]
	\centering
	\includegraphics[width=12cm]{img/schematics/keytag.jpg}
	\caption{Schemat modułu zasilania urządzenia deaktywującego. \\ Źródło: Opracowanie własne.}
	\label{fig:image_keytag_schematic}
\end{figure}
